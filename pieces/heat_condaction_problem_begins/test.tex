%%% Поля и разметка страницы %%%
\documentclass[a4paper,12pt]{article}
\usepackage{lscape}		% Для включения альбомных страниц

%%% Кодировки и шрифты %%%
\usepackage{cmap}						% Улучшенный поиск русских слов в полученном pdf-файле
\usepackage[T2A]{fontenc}				% Поддержка русских букв
\usepackage[utf8]{inputenc}				% Кодировка utf8
\usepackage[english, russian]{babel}	% Языки: русский, английский
%\usepackage{pscyr}						% Красивые русские шрифты

%%% Математические пакеты %%%
\usepackage{amsthm,amsfonts,amsmath,amssymb,amscd} % Математические дополнения от AMS

%%% Оформление абзацев %%%
\usepackage{indentfirst} % Красная строка

%%% Цвета %%%
\usepackage[usenames]{color}
\usepackage{color}
\usepackage{colortbl}

%%% Таблицы %%%
\usepackage{longtable}					% Длинные таблицы
\usepackage{multirow,makecell,array}	% Улучшенное форматирование таблиц

%%% Общее форматирование
\usepackage[singlelinecheck=off,center]{caption}	% Многострочные подписи
\usepackage{soul}									% Поддержка переносоустойчивых подчёркиваний и зачёркиваний

%%% Библиография %%%
\usepackage{cite} % Красивые ссылки на литературу

%%% Для многострочных формул gathered
\usepackage{amsmath}

%%% Для прямых, а не наклонных интегралов
\usepackage{wasysym}
\let\int\varint

%%% Гиперссылки %%%
\usepackage[plainpages=false,pdfpagelabels=false]{hyperref}
\definecolor{linkcolor}{rgb}{0.9,0,0}
\definecolor{citecolor}{rgb}{0,0.6,0}
\definecolor{urlcolor}{rgb}{0,0,1}
\hypersetup{
    colorlinks, linkcolor={linkcolor},
    citecolor={citecolor}, urlcolor={urlcolor}
}

%%% Изображения %%%
\usepackage{graphicx}		% Подключаем пакет работы с графикой
\graphicspath{{images/}}	% Пути к изображениям

%%% Выравнивание и переносы %%%
\sloppy					% Избавляемся от переполнений
\clubpenalty=10000		% Запрещаем разрыв страницы после первой строки абзаца
\widowpenalty=10000		% Запрещаем разрыв страницы после последней строки абзаца

%%% Библиография %%%
\makeatletter
\bibliographystyle{utf8gost705u}	% Оформляем библиографию в соответствии с ГОСТ 7.0.5
\renewcommand{\@biblabel}[1]{#1.}	% Заменяем библиографию с квадратных скобок на точку:
\makeatother

%%% Колонтитулы %%%
\let\Sectionmark\sectionmark
\def\sectionmark#1{\def\Sectionname{#1}\Sectionmark{#1}}
\makeatletter
\newcommand*{\currentname}{\@currentlabelname}
\renewcommand{\@oddhead}{\it \vbox{\hbox to \textwidth%
    % {\hfil Фамилия И.О. --- Короткое название черновика\hfil\strut}\hbox to \textwidth%
    {\today \hfil \thesection~\Sectionname\strut}\hrule}}
\makeatother

%%%%%%%%%%%%%%%%%%%%%%%%%%%%%%%%%%%%%%%%%%%%%%%%%%%%%%%%%%%%%%%%%%%%%%%%%%%%%%%%%%%
\begin{document}

\begin{center}
    \Huge{Асимптотическое расщепление задачи теплопроводности, начало}
\end{center}

\section{Введение}

Рассмотрим тело вырезанное из 3-периодисческой твердотельной среды (рис ) на которое действуют какие-либо тепловые нагрузки, тогда внутри тела должно выполнятся
стциотнарное уравнение теплового равновесия:
\begin{equation}
    \label{heat_equation}
    \frac{\partial q_x}{\partial x} + \frac{\partial q_y}{\partial y} + \frac{\partial q_z}{\partial z} = -Q
\end{equation}
где 
$Q$
- обьемные источниеки тепла, 
$q_\alpha$ 
- компоненты вектора теплового потока внутри среды.
На границеперехода от одной упругой среды к другой должны быть неперывны тепловой поток и температура:

\begin{equation}
    \label{hc_eq2}
    \left[ q_\alpha \right] = 0, \left[ T \right] = 0, \alpha \in \{ x,y,z \}
\end{equation}

Внутри среды возникает анизатропный закон Фурье, содержащий 6 независимых коэффициенотв теплопроводности 
$\lambda_{\alpha \beta}$:
\begin{equation}
    \label{hc_law}
    q_{\alpha} = - \sum_{\beta \in \{x,y,z\}}\lambda_{\alpha \beta}\frac{\partial T}{\partial \beta}, \alpha \in \{x,y,z\}
\end{equation}

Пусть 
$h$
- линейный размер пвериодической ячейки, 
$L$
- характерный размер тела, 
$T_{\ast}$ $\lambda_{\ast}$ 
- характерные значения температуры и коэффициента 
теплопроводности. Перрейдем к безразмерным переменным и функциям, для простоты не меняя обозначения:
\begin{equation}
x \leftrightarrow \frac{x}{L}, 
y \leftrightarrow \frac{y}{L}, 
z \leftrightarrow \frac{z}{L},
T \leftrightarrow \frac{T}{T_{\ast}},
\lambda_{\alpha \beta} \leftrightarrow \frac{\lambda_{\alpha \beta}}{\lambda_{\ast}},
q \leftrightarrow \frac{q}{q},
Q \leftrightarrow \frac{Q}{q}
\end{equation}

В дальнейщем будем считать, что отношение размера периодической ячейки упругой среды к характерному размеру тела является малым 
параметром и обозначается буквой 
$\varepsilon$:
\begin{equation}
    \varepsilon = \frac{h}{L} \ll 1
\end{equation}

Тогда уравнение 
(\ref{heat_equation}) 
и закон теплопроводности 
(\ref{hc_law}) 
в новых переменных примут вид:
\begin{equation}
    \frac{\partial q_x}{\partial x}\varepsilon + \frac{\partial q_y}{\partial y}\varepsilon + \frac{\partial q_z}{\partial z}\varepsilon = -Q
\end{equation}

\begin{equation}
    q_{\alpha} = \sum_{\beta \in \{x,y,z\}}\lambda_{\alpha\beta} \frac{\partial T}{\partial \beta}\varepsilon, \alpha \in \{x,y,z\}
\end{equation}

Внутри каждой периядической ячейке вводятся свои ячейковые координаты 
$\xi_x$, $\xi_y$, $\xi_z$:
\begin{equation}
    \label{cell_coor}
    \alpha=\alpha_i + \xi_{\alpha}\varepsilon, \; \xi_{\alpha} \in \{0,1\}, \; \alpha \in \{x,y,z\}
\end{equation}

Где - 
$\alpha_i$ 
координаты i-ой периодической ячейки. Коэффициэты теплопроводности 3-периодической среды являются функциями только ячейковых координат 
$\xi$:
\begin{equation}
    \lambda_{\alpha \beta} = \lambda_{\alpha \beta}(\xi_x, \xi_y, \xi_z)
\end{equation}

С учетом равеннства 
(\ref{cell_coor}) 
оператор частрого дифференциирования принимает вид:
\begin{equation}
    \label{diff_oper}
    \frac{\partial}{\partial \alpha}=\frac{\partial}{\partial \alpha} + \frac{1}{\varepsilon}\frac{\partial}{\partial \xi_{\alpha}}, \; \alpha \in \{x,y,z\}
\end{equation}

Задача 
(\ref{heat_equation})
-
(\ref{hc_law})
с учетом формулы 
(\ref{diff_oper})
принимает втд:
\begin{equation}
    \label{heat_equation_epsilon}
    \frac{q_x}{\partial x}\varepsilon+
    \frac{q_y}{\partial y}\varepsilon+
    \frac{q_z}{\partial z}\varepsilon+
    \frac{q_x}{\partial \xi_x}+
    \frac{q_y}{\partial \xi_y}+
    \frac{q_z}{\partial \xi_z}=
    -Q
\end{equation}

Условия на границе перехода от одной упругой среды к другой:
\begin{equation}
    \left[ q_\alpha \right] = 0, \left[ T \right] = 0, \alpha \in \{ x,y,z \}
\end{equation}

Закон теплопролводности:
\begin{equation}
    \label{hc_law_epsilon}
    q_{\alpha}=\sum_{\beta \in \{x,y,z\}}\lambda_{\alpha \beta}\left(\frac{\partial T}{\partial \beta}\varepsilon+\frac{\partial T}{\partial \xi_{\beta}}\right), \:
    \alpha \in \{x,y,z\}
\end{equation}

Для решения задачи 
(\ref{heat_equation_epsilon})
-
(\ref{hc_law_epsilon})
используем метод асимптотического расщепления. Для этого представим асимптотические приближения температуры и компонент теплового потока,
как суммы частных дифференциальных операторов, коэффициенты которых зависят только от ячейковых переменных:

\begin{equation}
    \label{asimp_T_q}
    \begin{aligned}
        T^{(n)} = 
        \sum^n_{k=0} \left( \sum_{k_x+k_y+k_z=k} \Psi^{\overline{k}}(\overline{\xi})  
        \frac{\partial^kT^{(n)}_0}{\partial \overline{r}^{\overline{k}}}\varepsilon^k \right) ,
        \\
        q^{(n)}_{\alpha} = 
        \sum^n_{k=0} \left( \sum_{k_x+k_y+k_z=k} K^{\overline{k}}_{\alpha}(\overline{\xi})  
        \frac{\partial^kT^{(n)}_0}{\partial \overline{r}^{\overline{k}}}\varepsilon^k \right)
    \end{aligned}
\end{equation}

где использованы следующие обозначения
\begin{equation}
    \begin{gathered}
        \partial \overline{r}^{\overline{k}} = \partial x^k \partial y^k \partial z^k, \;
        \overline{k} = (k_x,k_y,k_z) = k_x \overline{e}_x+k_y \overline{e}_y+k_z \overline{e}_z,
        \\
        |\overline{k}| = k = k_x+k_y+k_z, \; \overline{r} = (x.y.z) = x \overline{e}_x + y \overline{e}_y + z \overline{e}_z,
        \\
        \overline{\xi} = \left( \xi_x + \xi_y, + \xi_z \right) = \xi_x \overline{e}_x + \xi_y \overline{e}_y + \xi_z \overline{e}_z
    \end{gathered}
\end{equation}


Будем считать, что объёмные силы имеют расщепленный вид относительно переменных среды макросреды и ячейковых переменных:

\begin{equation}
    \label{volum_force}
    Q \left( \overline{r}, \overline{\xi} \right) = \kappa \left( \overline{\xi} \right) Q_0 \left( \overline{r} \right)
\end{equation}

причем равнодействующие сомножителя, зависят от быстрых переменных, на ячейке равняется еденице:

\begin{equation}
    \label{integ_vf}
    \int_0^1 \int_0^1 \int_0^1 \kappa \left( \overline{\xi} \right) d\xi_x d\xi_y d\xi_z = 1
\end{equation}

В дальнейшем интнграл от какой-то величины по ячейковым переменным, взятых по всей ячейке, будем называть усреднением этой величины по ячейке и обозначать:

\begin{equation}
    \left< \_ \right> = \int_0^1 \int_0^1 \int_0^1 \_ d\xi_x d\xi_y d\xi_z = 1
\end{equation}

А из равенства 
(\ref{volum_force})
-
(\ref{integ_vf})
следует что величина имеет физический смысл - это среднее значение теплового источника, т.е. 
это тепловой источник макросреды:
\begin{equation}
    Q_0 \left( \overline{r} \right) = \left<  Q \left( \overline{r}, \overline{\xi} \right) \right>
\end{equation}

Далее введем предположение, что тепловой источник макросреды также может быть разложен в степень дифференциальных операторов:
\begin{equation}
    \label{asimp_Q}
    Q_0 \left( \overline{r} \right) = \sum^n_{k=0} \left( \sum_{k_x+k_y+k_z=k} \Lambda^{ \overline{k}} \frac{\partial^k T^{(n)}_0}{\partial \overline{r}^{ \overline{k}}} \right)
    , \; \alpha \in \{x,y,z\}
\end{equation}

где 
$\Lambda^{ \overline{k}}$ 
- некоторые константы с векторним вкрхним индексом, которые будут определены позднее.

Подставим формулы 
(\ref{asimp_T_q})
,
(\ref{volum_force})
и 
(\ref{asimp_Q})
в равенства 
(\ref{heat_equation_epsilon})
-
(\ref{hc_law_epsilon}) и 
приравняем коэффициенты при одинакрвых дифференциальных операторах, получим систему уравнений в частных проиезводных
на неизвестные ячейковые функции:
уравнения теплового равновесия ячейки

\begin{equation}
    \label{eq_eq_cell}
    \frac{\partial K^{ \overline{k}}_x}{\partial \xi_x}+
    \frac{\partial K^{ \overline{k}}_y}{\partial \xi_y}+
    \frac{\partial K^{ \overline{k}}_z}{\partial \xi_z}+
    K^{ \overline{k}- \overline{e}_x}_x+
    K^{ \overline{k}- \overline{e}_y}_y+
    K^{ \overline{k}- \overline{e}_z}_z=
    -\kappa \left( \overline{\xi} \right) \Lambda^{ \overline{k}}
\end{equation}

закон теплопроводности внутри периодической ячейки

\begin{equation}
    K^{ \overline{k}}_{\alpha} \left( \xi \right) = 
    - \sum_{\beta \in \{x,y,z\}} \lambda_{\alpha\beta} 
    \left(  \frac{\partial \Psi^{ \overline{k}}}{\partial \xi_{\beta}} + \Psi^{ \overline{k} - \overline{e}_{\beta}}\right)
    , \; \alpha \in \{x,y,z\}
\end{equation}

условия сопряжения терловых потоков и температур внутри ячейки

\begin{equation}
    \left[  K^{ \overline{k}}_n \right] = 0, \; \left[  \Psi^{ \overline{k}}\right] = 0
\end{equation}

из условия периодичесности ячейковых функций

\begin{equation}
    \label{periodic_cell}
    \left. \Psi^{ \overline{k}} \left( \overline{\xi} \right) \right|_{\xi_{\alpha}=0} =
    \left. \Psi^{ \overline{k}} \left( \overline{\xi} \right) \right|_{\xi_{\alpha}=1}, \;
    \left. K^{ \overline{k}} \left( \overline{\xi} \right) \right|_{\xi_{\alpha}=0} =
    \left. K^{ \overline{k}} \left( \overline{\xi} \right) \right|_{\xi_{\alpha}=1}
    , \; \alpha \in \{x,y,z\}
\end{equation}

Выражение 
(\ref{eq_eq_cell})
-
(\ref{periodic_cell})
для каждого фиксированного целочисленного вектора 
$ \overline{k} $
представляют собой краевую эллиптическую задачу на нахождение 
периодических ячейковых функций 
$ \Psi^{ \overline{k}} $
. Необходимое условие разрешимости этой задачи имеет вид:

\begin{equation}
    \label{solv_cond}
    \Lambda^{ \overline{k}} = - \left< K^{ \overline{k} - \overline{e}_x}_x + K^{ \overline{k} - \overline{e}_y}_y + K^{ \overline{k} - \overline{e}_z}_z \right> 
\end{equation}

При 
$k=0$
решение задачи 
(\ref{eq_eq_cell})
-
(\ref{solv_cond})
имеет очевидное решение:

\begin{equation}
    \label{psi_one}
    \Psi^{ \overline{0}} = 1, ;\ K^{ \overline{0}}
    , \; \alpha \in \{x,y,z\}
\end{equation}

Тогда из 
(\ref{solv_cond})
следует равенство

\begin{equation}
    \label{lambda_zero}
    \Lambda^{ \overline{k}} = 0, \left| \overline{k} \right| 
\end{equation}

Равенство 
(\ref{asimp_Q})
с учетом равенста 
(\ref{lambda_zero})
имеет вид

\begin{equation}
    \label{asimp_Q_2}
    \sum^n_{k=2} \left( \sum_{k_x+k_y+k+z=k} \Lambda^{ \overline{k}} \frac{\partial^k T^{(n)_0}}{\partial \overline{r}^{ \overline{k}}} \right) =
    Q_0 \left( \overline{r} \right) 
\end{equation}

Данное равенство представляет соьбой уравнение на n-е приближение температуры макросреды 
$T^{(n)}_0$
. Уравнение представдялет собой уравнение в частных
производных, оно имеет порядок n, теория таких уравнений рассмотрена в 
\colorbox{yellow}{(ССЫЛКО!!!)}
, в частности, из нее следует, что асимптотический смысл имеет не 
все решения этого уравнения, а только часть из них, регулярно зависящая от малого параметра 
$\varepsilon$
, а это означат, что данное уравнение
имеет действительный порядок равный двум.
Формулы 
(\ref{asimp_T_q})
с учетом равенств 
(\ref{psi_one})
принимают вид

\begin{equation}
    \label{asimp_T_q_2}
    \begin{aligned}
        T^{(n)} = T^{(n)}_0 +
        \sum^n_{k=1} \left( \sum_{k_x+k_y+k_z=k} \Psi^{\overline{k}}(\overline{\xi})  
        \frac{\partial^kT^{(n)}_0}{\partial \overline{r}^{\overline{k}}}\varepsilon^k \right) ,
        \\
        q^{(n)}_{\alpha} = 
        \sum^n_{k=1} \left( \sum_{k_x+k_y+k_z=k} K^{\overline{k}}_{\alpha}(\overline{\xi})  
        \frac{\partial^kT^{(n)}_0}{\partial \overline{r}^{\overline{k}}}\varepsilon^k \right)
    \end{aligned}
\end{equation}

эти равенства являются формулами для вычисления температуры и компонент теплового потока в периодической среде на основе решения уравнения 
(\ref{asimp_Q_2})
и краевых задач 
(\ref{eq_eq_cell})
-
(\ref{solv_cond})
. Величина 
$T^{(n)}_0$ 
имеет физический смысл, она является средним значением распределения температуры на ячейке, т.е. эта величина является 
температурой однородной макросреды:

\begin{equation}
    T^{(n)}_0 = \left< T^{(n)} \left( \overline{r}, \overline{\xi} \right)  \right> 
\end{equation}

Наибольший интерес в любой асимптотической теории представляют самые первые приближения, в данном случае это 
$n=2$
. Усреднми вектор теплового потока 
(\ref{asimp_T_q_2})
при 
$n=2$
и рассмотрим его первое приближение, в дальнейшем верхние индексы, указывающие на номера асимптотического приближения, опускаем:

\begin{equation}
    \label{asimp_q_n_2}
    \widetilde{q}_{\alpha} = 
    \sum_{\phi \in \{x,y,z\}} \left< K^{ \overline{e}_{\phi}}_{\alpha} \left( \overline{\xi} \right)  \right> 
    \frac{\partial T_0}{\partial \phi} \varepsilon
    , \; \alpha \in \{x,y,z\}
\end{equation}

Можно показать, что этот вектор удовлетворяет следующему уравнению теплового баланса:

\begin{equation}
    \label{eq_eq_meta}
    \frac{\partial \widetilde{q}_x}{\partial x}\varepsilon + 
    \frac{\partial \widetilde{q}_y}{\partial y}\varepsilon + 
    \frac{\partial \widetilde{q}_z}{\partial z}\varepsilon = -Q
\end{equation}

В уравнении 
(\ref{eq_eq_meta}) 
спрва стоит тепловой источник макросреды, вектор 
$ \widetilde{q}_{\alpha}$ 
зависит только от переменных макросреды, поэтому можно считать, 
что вектор 
$ \widetilde{q}_{\alpha}$ 
- это вектор теплового потока в однородной макросреде, а уравнение 
(\ref{asimp_q_n_2}) 
- это закон теполопроводности в макросреде, это уравнение может быть переписано
в виде:

\begin{equation}
    \widetilde{q}_{\alpha} = - \sum_{ \phi \in \{x,y,z\} } \widetilde{\lambda}_{\alpha\phi} \frac{\partial T_0}{\partial \phi} \varepsilon
\end{equation}

где 
$ \widetilde{\lambda}_{\alpha\phi}$ 
- коэффициент теплопроводности макросреды, они рассчитываются на омнове решений данных ячейковых краевых задач:

\begin{equation}
    \widetilde{\lambda}_{\alpha\phi} = - \left< K^{ \overline{e}_{\phi}}_{\alpha} \left( \overline{\xi} \right) \right> 
    , \; \alpha,\phi \in \{x,y,z\} 
\end{equation}

Для расчета коэффициентов теплопроводности макросреды необходимо решить следующие три краевые задачи на ячейке:

урввнение
\begin{equation}
    \frac{K^{ \overline{e}_{\phi}}_x}{\partial \xi_x} +
    \frac{K^{ \overline{e}_{\phi}}_y}{\partial \xi_y} +
    \frac{K^{ \overline{e}_{\phi}}_z}{\partial \xi_z} = 0,
    \phi \in \{x,y,z\} 
\end{equation}

закон теплопроводности на ячейке
\begin{equation}
    K^{ \overline{e}_{\phi}}_{\alpha} \left( \overline{\xi} \right) =
    - \sum_{ \beta \in \{x,y,z\} } \lambda_{\alpha\beta}
    \left( \frac{\partial \Psi^{ \overline{e}_{\phi}}}{\partial \xi_{\beta}} + \delta^{\phi}_{\beta} \right) 
\end{equation}

условие непрерывности на границах раздела матрицы и включений
\begin{equation}
    \left[  K^{ \overline{e}_{\phi}}_n \right] = 0, \; \left[  \Psi^{ \overline{e}_{\phi}}\right] = 0
    , \; \phi \in \{x,y,z\}
\end{equation}

условие периодичности ячейковых функций
\begin{equation}
    \begin{gathered}
    \left. \Psi^{ \overline{e}_{\phi}} \left( \overline{\xi} \right) \right|_{\xi_{\alpha}=0} =
    \left. \Psi^{ \overline{e}_{\phi}} \left( \overline{\xi} \right) \right|_{\xi_{\alpha}=1}, \;
    \left. K^{ \overline{e}_{\phi}} \left( \overline{\xi} \right) \right|_{\xi_{\alpha}=0} =
    \left. K^{ \overline{e}_{\phi}} \left( \overline{\xi} \right) \right|_{\xi_{\alpha}=1},
    \\
    \alpha,\phi \in \{x,y,z\}
    \end{gathered}
\end{equation}




\end{document}
