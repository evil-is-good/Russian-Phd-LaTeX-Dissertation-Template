%%% Поля и разметка страницы %%%
\documentclass[a4paper,12pt]{article}
\usepackage{lscape}		% Для включения альбомных страниц

%%% Кодировки и шрифты %%%
\usepackage{cmap}						% Улучшенный поиск русских слов в полученном pdf-файле
\usepackage[T2A]{fontenc}				% Поддержка русских букв
\usepackage[utf8]{inputenc}				% Кодировка utf8
\usepackage[english, russian]{babel}	% Языки: русский, английский
%\usepackage{pscyr}						% Красивые русские шрифты

%%% Математические пакеты %%%
\usepackage{amsthm,amsfonts,amsmath,amssymb,amscd} % Математические дополнения от AMS

%%% Оформление абзацев %%%
\usepackage{indentfirst} % Красная строка

%%% Цвета %%%
\usepackage[usenames]{color}
\usepackage{color}
\usepackage{colortbl}

%%% Таблицы %%%
\usepackage{longtable}					% Длинные таблицы
\usepackage{multirow,makecell,array}	% Улучшенное форматирование таблиц

%%% Общее форматирование
\usepackage[singlelinecheck=off,center]{caption}	% Многострочные подписи
\usepackage{soul}									% Поддержка переносоустойчивых подчёркиваний и зачёркиваний

%%% Библиография %%%
\usepackage{cite} % Красивые ссылки на литературу

%%% Гиперссылки %%%
\usepackage[plainpages=false,pdfpagelabels=false]{hyperref}
\definecolor{linkcolor}{rgb}{0.9,0,0}
\definecolor{citecolor}{rgb}{0,0.6,0}
\definecolor{urlcolor}{rgb}{0,0,1}
\hypersetup{
    colorlinks, linkcolor={linkcolor},
    citecolor={citecolor}, urlcolor={urlcolor}
}

%%% Изображения %%%
\usepackage{graphicx}		% Подключаем пакет работы с графикой
\graphicspath{{images/}}	% Пути к изображениям

%%% Выравнивание и переносы %%%
\sloppy					% Избавляемся от переполнений
\clubpenalty=10000		% Запрещаем разрыв страницы после первой строки абзаца
\widowpenalty=10000		% Запрещаем разрыв страницы после последней строки абзаца

%%% Библиография %%%
\makeatletter
\bibliographystyle{utf8gost705u}	% Оформляем библиографию в соответствии с ГОСТ 7.0.5
\renewcommand{\@biblabel}[1]{#1.}	% Заменяем библиографию с квадратных скобок на точку:
\makeatother

%%% Колонтитулы %%%
\let\Sectionmark\sectionmark
\def\sectionmark#1{\def\Sectionname{#1}\Sectionmark{#1}}
\makeatletter
\newcommand*{\currentname}{\@currentlabelname}
\renewcommand{\@oddhead}{\it \vbox{\hbox to \textwidth%
    % {\hfil Фамилия И.О. --- Короткое название черновика\hfil\strut}\hbox to \textwidth%
    {\today \hfil \thesection~\Sectionname\strut}\hrule}}
\makeatother

%%%%%%%%%%%%%%%%%%%%%%%%%%%%%%%%%%%%%%%%%%%%%%%%%%%%%%%%%%%%%%%%%%%%%%%%%%%%%%%%%%%
\begin{document}

{~}\bigskip
\begin{center}
    \Huge{
    Конвертирование материальных констант 
$C_{ijkl}$ 
в технические: 
$E$, $\nu$
и пр.}
\end{center}

\section{Описание}
В обобщённом виде, для анизотропного матенриала, закон Гука имееет вид:
\begin{equation}
    \label{eq1}
    \sigma_{ij} = C_{ijkl}\varepsilon_{kl}
\end{equation}
Тензор 
$C_{ijkl}$ 
- тензор упругости составленный из материальных констант, всего их 81.
В силу симметрии тензоров 
$\varepsilon_{ij}$ 
и 
$\sigma_{ij}$:

$$
C_{ijkl} = C_{jikl} = C_{ijlk} = C_{jilk} 
$$

Количество независимых констант сокращается до 36. Далее из условия потенциальности тензора напряженияй:
$$
W=\frac{1}{2}C_{ijkl}\varepsilon_{ij}\varepsilon_{kl}
$$
Следует что 
$$
C_{ijkl} = C_{klij}
$$
Значит что всего независимых констант 21.

Обратное соотношение имеет вид:
$$
\varepsilon_{ij} = D_{ijkl}\sigma_{kl}
$$
Где 
$D$ 
- тензор податливости.

Записав тензоры 
$\varepsilon_{ij}$ 
и 
$\sigma_{ij}$ 
как вектора 
$\varepsilon_{\alpha}$ 
и
$\sigma_{\alpha}$
, где 
$\alpha \in \{xx, yy, zz, xy, xz, yz\}$
, представим соотношение \ref{eq1} в виде:
$$
    \sigma_{\alpha} = C_{\alpha\beta}\varepsilon_{\beta}
$$
А обратное:
$$
    \varepsilon_{\alpha} = D_{\alpha\beta}\sigma_{\beta}
$$
$$
D_{\alpha\beta} = C^{-1}_{\alpha\beta}
$$
Последнее, в более традиционной форме, записанное через технические константы, представляет собой такую систему:
\begin{equation*}
    \begin{cases}
        \varepsilon_{xx} = \frac{1}{E_x}\sigma_{xx} - \frac{\nu_{yx}}{E_y}\sigma_{yy} - \frac{\nu_{zx}}{E_z}\sigma_{zz} + \frac{\varkappa^1_{xyx}}{2G_{xy}}\sigma_{xy} + \frac{\varkappa^1_{xzx}}{2G_{xz}}\sigma_{xz} + \frac{\varkappa^1_{yzx}}{2G_{yz}}\sigma_{yz}\\
        \varepsilon_{yy} = - \frac{\nu_{xy}}{E_x}\sigma_{xx} + \frac{1}{E_y}\sigma_{yy} - \frac{\nu_{zy}}{E_z}\sigma_{zz} + \frac{\varkappa^1_{xyy}}{2G_{xy}}\sigma_{xy} + \frac{\varkappa^1_{xzy}}{2G_{xz}}\sigma_{xz} + \frac{\varkappa^1_{yzy}}{2G_{yz}}\sigma_{yz}\\
        \varepsilon_{xx} = - \frac{\nu_{xz}}{E_x}\sigma_{xx} - \frac{\nu_{yz}}{E_y}\sigma_{yy} + \frac{1}{E_z}\sigma_{zz} + \frac{\varkappa^1_{xyx}}{2G_{xz}}\sigma_{xy} + \frac{\varkappa^1_{xzz}}{2G_{xz}}\sigma_{xz} + \frac{\varkappa^1_{yzz}}{2G_{yz}}\sigma_{yz}\\
        \varepsilon_{xy} = \frac{\varkappa^2_{xxy}}{E_x}\sigma_{xx} + \frac{\varkappa^2_{yxy}}{E_y}\sigma_{yy} + \frac{\varkappa^2_{zxy}}{E_z}\sigma_{zz} + \frac{1}{2G_{xy}}\sigma_{xy} + \frac{\eta_{xzxy}}{2G_{xz}}\sigma_{xz} + \frac{\eta_{yzxy}}{2G_{yz}}\sigma_{yz}\\
        \varepsilon_{xz} = \frac{\varkappa^2_{xxz}}{E_x}\sigma_{xx} + \frac{\varkappa^2_{yxz}}{E_y}\sigma_{yy} + \frac{\varkappa^2_{zxz}}{E_z}\sigma_{zz} + \frac{\eta_{xyxz}}{2G_{xy}}\sigma_{xy} + \frac{1}{2G_{xz}}\sigma_{xz} + \frac{\eta_{yzxz}}{2G_{yz}}\sigma_{yz}\\
        \varepsilon_{xz} = \frac{\varkappa^2_{xyz}}{E_x}\sigma_{xx} + \frac{\varkappa^2_{yyz}}{E_y}\sigma_{yy} + \frac{\varkappa^2_{zyz}}{E_z}\sigma_{zz} + \frac{\eta_{xyyz}}{2G_{xy}}\sigma_{xy} + \frac{\eta_{xzyz}}{2G_{xz}}\sigma_{xz} + \frac{1}{2G_{yz}}\sigma_{yz}
    \end{cases}
\end{equation*}
Здесь:

$E_i$ - модули Юнга, модули нормальной упругости определяющие величину линейной деформации в направлении 
$i$ (то есть деформации 
$\varepsilon_{ii}$)
при действии одних только нормальных напряжений в этом же направлении (то есть напряжений 
$\sigma_{ii}$);

$\nu_{ij}$ - коэффициенты Пуассона, определяющие величину линейной деформации в направлении 
$j$ 
(то есть деформации 
$\sigma_{jj}$)
при действии одних только нормальных напряжений в направлении 
$i$ 
(то есть напряжений 
$\sigma_{ii}$);

$G_{ij}$ - модули сдвига для плоскостей определяющие величину сдвиговой деформации в плоскости 
$ij$ 
(то есть деформации 
$\varepsilon_{ij}$)
при действии одних только касательных напряжений в этой же плоскости (то есть напряжений 
$\sigma_{ij}$);

$\varkappa^1_{kli}$ - коэффициенты взаимного влияния, определяющие величину линейной деформации в направлении 
$i$ 
(то есть деформации 
$\varepsilon_{ii}$)
при действии одних только касательных напряжений в плоскости 
$kl$ 
(то есть напряжений 
$\sigma_{kl}$);

$\varkappa^2_{ikl}$ - коэффициенты взаимного влияния, определяющие величину сдвиговой деформации в плоскости 
$kl$ 
(то есть деформации 
$\varepsilon_{kl}$)
при действии одних только нормальных напряжений в направлении 
$i$ 
(то есть напряжений 
$\sigma_{ii}$);

Таким образом для того чтобы из тензора упругости 
$C_{\alpha\beta}$ 
получить технические константы необходимо получить обратную матрицуш 
$D_{\alpha\beta}$ 
и из неё уже извлечь эти константы.

$\eta_{ijkl}$ - коэффициенты Ченцова, определяющие величину сдвиговой деформации в плоскости 
$kl$ 
(то есть деформации 
$\varepsilon_{kl}$)
при действии одних только касательных напряжений в плоскости 
$ij$ 
(то есть напряжений 
$\sigma_{ij}$);

\section{
Аналитические формулы для преобразования в 
$E$ 
и 
$\nu$}

$$
E_x = \frac{A}{C_{yyyy}C_{zzzz}-C_{yyzz}^2}
$$

$$
E_y = \frac{A}{C_{xxxx}C_{zzzz}-C_{xxzz}^2}
$$

$$
E_z = \frac{A}{C_{xxxx}C_{yyyy}-C_{xxyy}^2}
$$

$$
A = C_{xxxx}C_{yyyy}C_{zzzz} - C_{xxxx}C_{yyzz}^2 - C_{yyyy}C_{xxzz}^2 -
C_{zzzz}C_{xxyy}^2 + 2C_{xxyy}C_{yyzz}C_{xxzz}
$$

$$
\nu_{xy} = \frac{C_{xxyy}C_{zzzz}-C_{xxzz}C_{yyzz}}{C_{yyyy}C_{zzzz}-C_{yyzz}^2}
$$

$$
\nu_{xz} = \frac{C_{xxzz}C_{yyyy}-C_{xxyy}C_{yyzz}}{C_{yyyy}C_{zzzz}-C_{yyzz}^2}
$$

$$
\nu_{yx} = \frac{C_{xxyy}C_{zzzz}-C_{xxzz}C_{yyzz}}{C_{xxxx}C_{zzzz}-C_{xxzz}^2}
$$

$$
\nu_{yz} = \frac{C_{yyzz}C_{xxxx}-C_{xxyy}C_{xxzz}}{C_{xxxx}C_{zzzz}-C_{xxzz}^2}
$$

$$
\nu_{zx} = \frac{C_{xxzz}C_{yyyy}-C_{xxyy}C_{yyzz}}{C_{xxxx}C_{yyyy}-C_{xxyy}^2}
$$

$$
\nu_{zy} = \frac{C_{yyzz}C_{xxxx}-C_{xxyy}C_{xxzz}}{C_{xxxx}C_{yyyy}-C_{xxyy}^2}
$$




\end{document}
