Эти наблюдения, сделанные Хиллом (1963a), дают строгие определения макро модулей, для одного конкретного образца.
Модули относятся к тестам, которые фактически могут быть выполнены для образца изготовленного из такого материала, и применяются независимо от того, и применимы для произвольных граничных условий. 
В самом деле, две отдельные краевые задачи (1) и (2), дали бы разные значения для макромодуля, если бы они применялись к произвольной неоднородной среде.
Это вомпринимается очень вероятным, а не строго установленно что уравнение (I) устанавливает <<среднее>> поведение композита в случае произвольных граничных условий.
Случаи, когда оценки модуля по (1) и (2) согласуются можно считать как частичную проверку концепции <<макромодуля>>.
В случае периодического композита вдали от пограничного слоя эти оценки согласуются и даюте периодические поля для напряжений и деформаций с одинаковыми средними значениями.
Санчес-Паленсия определял понятие среднего модуля именно для периодических сред по одной ячейке.
 Выражение (2.23) важно тем что озволяет определить границы возможных значений для модуля. 
 Так как при заданных на границах перемещениях, поле внутри теля является минимально возможным, является минимизацией потенцеала энергии.
