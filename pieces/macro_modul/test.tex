%%% Поля и разметка страницы %%%
\documentclass[a4paper,12pt]{article}
\usepackage{lscape}		% Для включения альбомных страниц

%%% Кодировки и шрифты %%%
\usepackage{cmap}						% Улучшенный поиск русских слов в полученном pdf-файле
\usepackage[T2A]{fontenc}				% Поддержка русских букв
\usepackage[utf8]{inputenc}				% Кодировка utf8
\usepackage[english, russian]{babel}	% Языки: русский, английский
%\usepackage{pscyr}						% Красивые русские шрифты

%%% Математические пакеты %%%
\usepackage{amsthm,amsfonts,amsmath,amssymb,amscd} % Математические дополнения от AMS

%%% Оформление абзацев %%%
\usepackage{indentfirst} % Красная строка

%%% Цвета %%%
\usepackage[usenames]{color}
\usepackage{color}
\usepackage{colortbl}

%%% Таблицы %%%
\usepackage{longtable}					% Длинные таблицы
\usepackage{multirow,makecell,array}	% Улучшенное форматирование таблиц

%%% Общее форматирование
\usepackage[singlelinecheck=off,center]{caption}	% Многострочные подписи
\usepackage{soul}									% Поддержка переносоустойчивых подчёркиваний и зачёркиваний

%%% Библиография %%%
\usepackage{cite} % Красивые ссылки на литературу

%%% Для многострочных формул gathered
\usepackage{amsmath}

%%% Для прямых, а не наклонных интегралов
\usepackage{wasysym}
\let\int\varint

%%% Гиперссылки %%%
\usepackage[plainpages=false,pdfpagelabels=false]{hyperref}
\definecolor{linkcolor}{rgb}{0.9,0,0}
\definecolor{citecolor}{rgb}{0,0.6,0}
\definecolor{urlcolor}{rgb}{0,0,1}
\hypersetup{
    colorlinks, linkcolor={linkcolor},
    citecolor={citecolor}, urlcolor={urlcolor}
}

%%% Изображения %%%
\usepackage{graphicx}		% Подключаем пакет работы с графикой
\graphicspath{{images/}}	% Пути к изображениям

%%% Выравнивание и переносы %%%
\sloppy					% Избавляемся от переполнений
\clubpenalty=10000		% Запрещаем разрыв страницы после первой строки абзаца
\widowpenalty=10000		% Запрещаем разрыв страницы после последней строки абзаца

%%% Библиография %%%
\makeatletter
\bibliographystyle{utf8gost705u}	% Оформляем библиографию в соответствии с ГОСТ 7.0.5
\renewcommand{\@biblabel}[1]{#1.}	% Заменяем библиографию с квадратных скобок на точку:
\makeatother

%%% Колонтитулы %%%
\let\Sectionmark\sectionmark
\def\sectionmark#1{\def\Sectionname{#1}\Sectionmark{#1}}
\makeatletter
\newcommand*{\currentname}{\@currentlabelname}
\renewcommand{\@oddhead}{\it \vbox{\hbox to \textwidth%
    % {\hfil Фамилия И.О. --- Короткое название черновика\hfil\strut}\hbox to \textwidth%
    {\today \hfil \thesection~\Sectionname\strut}\hrule}}
\makeatother

%%%%%%%%%%%%%%%%%%%%%%%%%%%%%%%%%%%%%%%%%%%%%%%%%%%%%%%%%%%%%%%%%%%%%%%%%%%%%%%%%%%
\begin{document}
\section{Что такое композиты}
Эти наблюдения, сделанные Хиллом (1963a), дают строгие определения макро модулей, для одного конкретного образца.
Модули относятся к тестам, которые фактически могут быть выполнены для образца изготовленного из такого материала, и применяются независимо от того, и применимы для произвольных граничных условий. 
В самом деле, две отдельные краевые задачи (1) и (2), дали бы разные значения для макромодуля, если бы они применялись к произвольной неоднородной среде.
Это вомпринимается очень вероятным, а не строго установленно что уравнение (I) устанавливает <<среднее>> поведение композита в случае произвольных граничных условий.
Случаи, когда оценки модуля по (1) и (2) согласуются можно считать как частичную проверку концепции <<макромодуля>>.
В случае периодического композита вдали от пограничного слоя эти оценки согласуются и даюте периодические поля для напряжений и деформаций с одинаковыми средними значениями.
Санчес-Паленсия определял понятие среднего модуля именно для периодических сред по одной ячейке.
 Выражение (2.23) важно тем что озволяет определить границы возможных значений для модуля. 
 Так как при заданных на границах перемещениях, поле перемещений внутри теля является минимально возможным, то оно является минимизацией потенцеала энергии (2.27).

 Наблюдение, что оценки Фойгта и Ройсса дают оценки для компонентов E, принадлежит Хиллу (1952).

 Если образец, на который накладываются граничные условиия рассматривается как часть ансабля. 
 Напряжения, девормации и перемещения зависят от некоторого параметра альфа.
 И можно найти их средние значения.
 Тогда в каждой точке можно определить макромодуль как (2.35)






\end{document}
