Рассмотрим тело вырезанное из 3-периодической среды, на которое действуют какиелибо нагрузки, тогда внутри тела должны ваполнятся уравнения
равновесия:

\begin{equation}
    \label{elhp:eq1}
    \frac{\partial \sigma_{\beta x}}{\partial x} +
    \frac{\partial \sigma_{\beta y}}{\partial y} +
    \frac{\partial \sigma_{\beta z}}{\partial z} +
    F_{\beta} = 0,
    \beta \in \{x,y,z\} 
\end{equation}

гда 
$F_{\beta}$ 
- обьёмные силы. На границе перехода от одной упругой чреды к другой должны быть непрорывны перемещения и контактные напряжения:

\begin{equation}
    \label{elhp:eq2}
    \left[ \sigma_{\beta n} \right] = 0, \left[ u_{\beta} \right] = 0, \beta \in \{x,y,z\} 
\end{equation}

где $\sigma_{\beta n}$ -  контактные напряжения, которые по определению вычисляются по следующей формуле

\begin{equation}
    \label{elhp:eq3}
    \sigma_{\beta n} = 
    \sigma_{\beta x} n_x +
    \sigma_{\beta y} n_y +
    \sigma_{\beta z} n_z
\end{equation}

Считаем, что каждый материал является анизотропно уругим. Закон Дюамеля-Неймона зависимости напряжений от температуры в общем случае для
каждого материала содержит 21 независимую упрегую константц 
$E_{\gamma\beta\phi\psi}$ 
и 6 констант тензора линейного температурного 
расширения 
$\alpha_{\phi\psi}$ 
и имеет вид (\colorbox{yellow}{ССЫЛКО!!!})

\begin{equation}
    \label{elhp:eq4}
    \sigma_{\gamma \beta} = \sum_{ \phi, \psi \in \{x,y,z\} } E_{\gamma\beta\phi\psi} 
    \left( e_{\phi\psi} - \alpha_{\phi\psi}T \right) \gamma,\beta \in \{x,y,z\} 
\end{equation}

Компоненты тензора деформаций связаны с компонентами вектора перемещений соотношениями Коши:

\begin{equation}
    \label{elhp:eq5}
    e_{\gamma\beta} = \frac{1}{2} \left( \frac{\partial u_{\gamma}}{\partial \beta} + \frac{\partial u_{\beta}}{\partial \gamma}  \right) 
    ,\; \gamma,\beta \in \{x,y,z\} 
\end{equation}

Кроме того, к равенству 
(\ref{elhp:eq1})
-
(\ref{elhp:eq5}) 
должны быть добавлены краевые условия на границе тела. 
С учетом их, задача 
(\ref{elhp:eq1})
-
(\ref{elhp:eq5}) 
является краевой задачей, решениями 
которой явзяются перемещения, по которым их равенства 
(\ref{elhp:eq4}) 
могут быть найдены термонапряжения.

Пусть i
$ \widetilde{u} $ 
- характерное значение для перемещения 
$h_x$, $h$ 
- линейный размер периодической решетки (куба), 
$L$ 
- характерный размер тела,
$ \widetilde{E}$, $ \widetilde{T}$ 
- характерное среднее значение модуля Юнга и изменения температуры соответственно. Перейдём к безразмерным переменным и функциям,
для простоты не меняя их обозначения:

\begin{equation}
    \label{elhp:eq6}
    \begin{aligned}
        x \leftrightarrow \frac{x}{L}, 
        y \leftrightarrow \frac{y}{L}, 
        z \leftrightarrow \frac{z}{L},
        u_{\gamma} \leftrightarrow \frac{u_{\gamma}}{h},
        E_{\gamma\beta\phi\psi} \leftrightarrow \frac{E_{\gamma\beta\phi\psi}}{ \widetilde{E}},
        \\
        \sigma_{\gamma\beta}\leftrightarrow \frac{\sigma_{\gamma\beta}}{ \widetilde{E}},
        q_{\gamma} \leftrightarrow \frac{q_{\gamma}}{ \widetilde{E}},
        T \leftrightarrow \frac{T}{ \widetilde{T}},
        \alpha_{\phi\psi} \leftrightarrow \alpha_{\phi\psi} \widetilde{T},
        \widetilde{F}_{\gamma} \leftrightarrow \frac{F_{\gamma}}{ \widetilde{E}}
    \end{aligned}
\end{equation}

В дальнейшем будем считать, что отношение размера периодической ячейки упругой среды к характерному размеру тела является малым параметром
и обознчается буквой 
$\varepsilon$:

\begin{equation}
    \label{elhp:eq7}
    \varepsilon = \frac{h}{L} \ll 1
\end{equation}

Тогда уравнения 
(\ref{elhp:eq1})
в новых переменных примут вид

\begin{equation}
    \label{elhp:eq8}
    \frac{\partial \sigma_{\gamma x}}{\partial x}\varepsilon +
    \frac{\partial \sigma_{\gamma y}}{\partial y}\varepsilon +
    \frac{\partial \sigma_{\gamma z}}{\partial z}\varepsilon +
    F_{\alpha} = 0
\end{equation}

Равенства 
(\ref{elhp:eq2})
-
(\ref{elhp:eq4})
останутся без изменений, а выражение 
(\ref{elhp:eq5})
примет вид

\begin{equation}
    \label{elhp:eq9}
    e_{\gamma\beta} = \frac{1}{2}
    \left( \frac{\partial u_{\gamma}}{\partial \beta} \varepsilon \frac{\partial u_{\beta}}{\partial \gamma} \varepsilon \right) 
\end{equation}

Введение двух масштабов моделирования. Согласно основной идее многомасштабного метода (\colorbox{yellow}{ССЫЛКО!!!}) внутри каждой периодической ячецки положение
точки однозначно описывается тремя безразмерными координатами 
$\xi_x, \xi_y, \xi_z$
, с помощью формул

\begin{equation}
    \label{elhp:eq10}
    x = x_i + \xi_x \varepsilon ,\;
    y = y_i + \xi_y \varepsilon ,\;
    z = z_i + \xi_z \varepsilon ,\;
    \xi_x,\xi_y,\xi_z \in \{x,y,z\} 
\end{equation}

где
$x_i, y_i, z_i$ 
- координаты i-го периодического куба. Координаты  
$\xi_x, \xi_y, \xi_z$ 
будем называть ячейквыми в противоположность пространственным 
координатам 
$x,y,z$
, в то время когда ячейковые координаты изменяются на единицу, соответствующие им простренственные меняются на малую величину 
$\varepsilon$
, как это следует из формулы 
(\ref{elhp:eq10})
.

Упругие и термоупругие характеристики 3-периодической среды будем считать зависящими только от ячейковых координат 
$\varepsilon$:

\begin{equation}
    \label{elhp:eq11}
    E_{\gamma\beta\phi\psi} = 
    E_{\gamma\beta\phi\psi} 
    \left( \overline{\xi} \right) 
    ,\;
    \alpha_{\phi\psi} = 
    \alpha_{\phi\psi}
    \left( \overline{\xi} \right) 
\end{equation}

Хотя солгасно формулам 
(\ref{elhp:eq10})
координаты i
$ \left( x,y,z \right) $ 
и 
$ \left( \xi_x, \xi_y, \xi_z \right) $ 
являются зависимыми, в дальнейшем
будем формально считать, что все неизвестные функции
одновременно зависят как от ячейковых координат 
$ \left( \xi_x, \xi_y, \xi_z \right) $
, так и от координат 
$ \left( x,y,z \right) $

\begin{equation}
    \label{elhp:eq13}
    u_{\gamma} =
    u_{\gamma}
    \left( \overline{\xi}, \overline{r} \right) 
    ,\;
    \sigma_{\gamma\beta} = 
    \sigma_{\gamma\beta}
    \left( \overline{\xi}, \overline{r} \right) 
    ,\;
    \gamma,\beta \in \{x,y,z\} 
\end{equation}

при этом фломулы 
(\ref{elhp:eq10})
продолжют остраваться справедливыми. Координаты 
$ \left( x,y,z \right) $ 
в дальнейшем будем называть макрокоординатами (макропеременными)
так как они однозначным образом фиксируют положение макросреды, которая будет введена в дальнейшем. Потребуем также условия периодичности неизвестных
функций по ячейковым перемещениям и напряжениям:

\begin{equation}
    \label{elhp:eq14}
    \begin{gathered}
    u_{\beta} \left. \left( \overline{\xi}, \overline{r} \right)  \right|_{\xi_{\gamma}=0} =
    u_{\beta} \left. \left( \overline{\xi}, \overline{r} \right)  \right|_{\xi_{\gamma}=1} 
    ,\;
    \sigma_{\gamma\beta} \left. \left( \overline{\xi}, \overline{r} \right)  \right|_{\xi_{\gamma}=0} =
    \sigma_{\gamma\beta} \left. \left( \overline{\xi}, \overline{r} \right)  \right|_{\xi_{\gamma}=1} 
    ,
    \\
    \gamma,\beta \in \{x,y,z\} 
    \end{gathered}
\end{equation}

С учеотом равенства 
(\ref{elhp:eq10})
оператор частного дифференцирования принимает вид диыыеренциирования и по ячейковым переменным и по макропеременнвм (\colorbox{yellow}{ССЫЛКО!!!}):

\begin{equation}
    \label{elhp:eq15}
    \frac{\partial }{\partial \gamma} = \frac{\partial }{\partial \gamma} + \frac{1}{\varepsilon} \frac{\partial \xi_{\varepsilon}}{\partial } 
\end{equation}

Уравнения 
(\ref{elhp:eq8})
, 
(\ref{elhp:eq3})
-
(\ref{elhp:eq5})
, 
(\ref{elhp:eq10})
с учетом формулы 
(\ref{elhp:eq15})
для диференциального оператора принимают вид

\begin{equation}
    \label{elhp:eq16}
    \frac{\partial \sigma_{\gamma x}}{\partial x} \varepsilon + 
    \frac{\partial \sigma_{\gamma y}}{\partial y} \varepsilon + 
    \frac{\partial \sigma_{\gamma z}}{\partial z} \varepsilon + 
    \frac{\partial \sigma_{\gamma x}}{\partial \xi_x} \varepsilon + 
    \frac{\partial \sigma_{\gamma y}}{\partial \xi_y} \varepsilon + 
    \frac{\partial \sigma_{\gamma z}}{\partial \xi_z} \varepsilon + 
    F_{\gamma} = 0
\end{equation}

условие на границе от одной упругой среды к другой:

\begin{equation}
    \label{elhp:eq17}
    \left[ \sigma_{\gamma n} \right] = 0, \; \left[ u_{\gamma} \right], \; \gamma,\beta \in \{x,y,z\} 
\end{equation}

закон Дюамеля-Неймона:

\begin{equation}
    \label{elhp:eq18}
    \sigma_{\gamma\beta} = \sum_{ \phi\psi \in \{x,y,z\} }
    E_{\gamma\beta\phi\psi} 
    \left( e_{\phi\psi} - \alpha_{\phi\psi}T \right) 
    , \;
    \gamma,\beta \in \{x,y,z\} 
\end{equation}

формулы для компонента тензора деформаций:

\begin{equation}
    \label{elhp:eq19}
    e_{\gamma\beta} = \frac{1}{2}
    \left( 
        \frac{\partial u_{\gamma}}{\partial \beta} \varepsilon +
        \frac{\partial u_{\gamma}}{\partial \beta} \varepsilon +
        \frac{\partial u_{\gamma}}{\partial \xi_{\beta}} +
        \frac{\partial u_{\gamma}}{\partial \xi_{\beta}}
    \right) 
    , \;
    \gamma,\beta \in \{x,y,z\} 
\end{equation}

Кроме того, к уравнениям 
(\ref{elhp:eq15})
-
(\ref{elhp:eq19})
должны быть добавлены краевые условия на кганице тела.
Далее будем считать, что законы ремпределения температуры и обэёмных сил по объйму среды имеют расщепленый вид относительно  медленных и быстрых
переменных:

\begin{equation}
    \label{elhp:eq20}
    T \left( \overline{r}, \overline{\xi} \right) = \Psi \left( \overline{\xi}  \right) T_0 \left( \overline{r}  \right) 
    , \;
    F_{\gamma} \left( \overline{r}, \overline{\xi} \right) = q_{\gamma} \left( \overline{\xi}  \right) f_{\gamma} \left( \overline{r}  \right) 
\end{equation}

причем интеграл сомножтеле, зависящих от быстрых переменных, по объёму периодический ячейки равняется единице:

\begin{equation}
    \label{elhp:eq21}
    \left< \Psi \left( \overline{\xi}  \right)  \right> = 1
    ,\;
    \left< q_{\gamma} \left( \overline{\xi}  \right)  \right> = 1
\end{equation}

Условие 
(\ref{elhp:eq21})
является нормировочным. А из равенств 
(\ref{elhp:eq19})
-
(\ref{elhp:eq21})
следует, функция 
$T_0 \left(  \overline{r} \right) $ 
является средней температурой в 
ячейке периодичности, а 
$f_{_gamma} \left( \overline{r}  \right) $
- средней объёмной силой, действующей на ячейку периодичности: 

\begin{equation}
    \label{elhp:eq22}
    T_0 \left( \overline{r}  \right) = \left< T \left( \overline{r} , \overline{\xi}  \right)  \right> 
    ,\;
    f_{\gamma} \left( \overline{r}  \right) = \left< T \left( \overline{r} , \overline{\xi}  \right)  \right> 
    ,\;
    \gamma \in \{x,y,z\} 
\end{equation}

Метод асимптотического расщепления. В соответствии с общей идеей метода асимптотического расщепления ищем перемещения и напряжения в виде
конечных сумм дифференнциальных операторов по макропеременным, а среднюю температуру и средние объёмные силы также представляем в виде суммы 
дифференциированных операторов по макропеременным, а среднюю температуру и средние объёмные силы также представляем в виде сумм дифференциальных 
операторов по макропеременным, 
$n$
- номер асимаптотического приближения:

\begin{equation}
    \label{elhp:eq23}
    \left( u_{\gamma} \right)^{(n)} =
    v_{\gamma}^{(n)} +
    \sum_{ \omega \in \{x,y,z\} } \sum_{k=1}^{n+1}
    \left( 
    \sum_{k_x+k_y+k_z = k} 
    \left( U_{\gamma}^{v_{\theta}} \right)^{ \overline{k} } 
    \frac{\partial^k v_{\theta}^{(n)}}{\partial \overline{r}^{ \overline{k}  } } 
    \varepsilon^k
\right) 
,\;
\gamma \in \{x,y,z\} 
\end{equation}

\begin{equation}
    \label{elhp:eq24}
    \left( \sigma_{\gamma\beta} \right)^{(n)} =
    \sum_{ \omega \in \{x,y,z\} } \sum_{k=1}^{n+1}
    \left( 
    \sum_{k_x+k_y+k_z = k} 
    \left( \tau_{ \gamma\beta}^{v_{\theta}} \right)^{ \overline{k} }
    \frac{\partial^k v_{\theta}^{(n)}}{\partial \overline{r}^{ \overline{k}  } } 
    \varepsilon^k
    \right) 
,\;
    \gamma,\beta \in \{x,y,z\} 
\end{equation}

\begin{equation}
    \label{elhp:eq25}
    T_0 \left( \overline{r}  \right) =
    \sum_{ \omega \in \{x,y,z\} } 
    \left( 
        \sum_{k=1}^{n}
        A_{v_{\theta}}^{ \overline{k} }
        \frac{\partial^k v_{\theta}^{(n)}}{\partial \overline{r}^{ \overline{k}  } } 
        \varepsilon^k
    \right) 
\end{equation}

\begin{equation}
    \label{elhp:eq26}
    f_{ \gamma} \left( \overline{r}  \right) =
    \sum_{ \omega \in \{x,y,z\} } \sum_{k=2}^{n+1}
    \left( 
    \sum_{k_x+k_y+k_z = k} 
    \left( B_{\gamma}^{v_{\theta}} \right)^{ \overline{k} } 
    \frac{\partial^k v_{\theta}^{(n)}}{\partial \overline{r}^{ \overline{k}  } } 
    \varepsilon^k
\right) 
,\;
\gamma \in \{x,y,z\} 
\end{equation}


где функции 
$v_{ \gamma}^{(n)}$ 
зависят от макропеременных и поопределению равняются осреднению реальных перемещений 
$ \left(  u_{ \gamma} \right)^{(n)} $ 
по периодической ячейке:

\begin{equation}
    \label{elhp:eq27}
    v_{ \gamma}^{(n)} = \left< \left( u_{ \gamma}^{(n)} \right)  \right> 
\end{equation}

В формулах 
(\ref{elhp:eq23})
-
(\ref{elhp:eq24})
кохффициенты при степенях дифференциальных операторов по макроперемещениым зависят только от ячейковых переменных, 
поэтому их называют ячейковыми функсциями. Соответственно и метод ассимптотического расщепления применительно к задачам такого типа
называются меотдом ячейковых функций. В формулах 
(\ref{elhp:eq25})
-
(\ref{elhp:eq26})
коэффициенты являются неизвестными константами. Для рещения краевой задачи

(\ref{elhp:eq16})
-
(\ref{elhp:eq19})
достаточно подставить в нее равенмтва 
(\ref{elhp:eq23})
-
(\ref{elhp:eq26})
, затем собрать подобные и приравнять коэффициенты при дифференциальных операторах 
нулю, в итоге получается система краевых рекурентных задач на неизвестных ячейковые функции вектора перемещений и тензора напряжений:

\begin{equation}
    \label{elhp:eq28}
    \frac{\partial \left(  \tau_{ \gamma x}^{v_{\theta}}\right)^{ \overline{k} }}{\partial \xi_x} +
    \frac{\partial \left(  \tau_{ \gamma y}^{v_{\theta}}\right)^{ \overline{k} }}{\partial \xi_y} +
    \frac{\partial \left(  \tau_{ \gamma z}^{v_{\theta}}\right)^{ \overline{k} }}{\partial \xi_z} +
    \left(  \tau_{ \gamma x}^{v_{\theta}}\right)^{ \overline{k} - \overline{\epsilon_x} } +
    \left(  \tau_{ \gamma y}^{v_{\theta}}\right)^{ \overline{k} - \overline{\epsilon_y} } +
    \left(  \tau_{ \gamma z}^{v_{\theta}}\right)^{ \overline{k} - \overline{\epsilon_z} } =
    q_{\gamma} \left( \overline{\xi}  \right) \left( B_{\gamma}^{v_{\theta}} \right)^{ \overline{k} }
\end{equation}

закон термоупругости внутри ячейки

\begin{equation}
    \label{elhp:eq29}
    \begin{gathered}
    \left( \tau_{ \gamma\beta} \right)^{ \overline{k} } =
    \sum_{ \phi,\psi \in \{x,y,z\}} E_{ \gamma\beta\phi\psi}
    \\
    \left( \frac{1}{2} 
        \left( 
            \frac{\partial \left( U_{\phi}^{v_{\theta}}\right)^{ \overline{k} }}{\partial \xi_{\psi}} +
            \frac{\partial \left( U_{\psi}^{v_{\theta}}\right)^{ \overline{k} }}{\partial \xi_{\phi}} +
            \left( U_{\phi}^{v_{\theta}}\right)^{ \overline{k} - \overline{\epsilon}_{\psi}  } +
            \left( U_{\psi}^{v_{\theta}}\right)^{ \overline{k} - \overline{\epsilon}_{\phi}  }
        \right) 
        - \alpha_{\phi\psi} \Psi A_{v_{\theta}}^{ \overline{k} }
    \right) 
    \end{gathered}
\end{equation}

условия непрерывности ячейковых функций внутри ячейке на границе различных сред

\begin{equation}
    \label{elhp:eq30}
    \left[ \left( \tau_{ \gamma n}^{v_{\theta}} \right)^{ \overline{k} }  \right] = 0
    ,\;
    \left[ \left( U_{ \gamma}^{v_{\theta}} \right)^{ \overline{k} }  \right] = 0
    ,\;
    \gamma \in \{x,y,z\} 
\end{equation}

услоние периодичности ячейковых функций

\begin{equation}
    \label{elhp:eq70}
    \begin{gathered}
    \left.   
    \left( U_{\beta}^{ v_{\theta}} \right)^{ \overline{k}}
    \left( \overline{\xi}  \right) 
    \right|_{\xi_{\gamma}=0}
    =
    \left.   
    \left( U_{\beta}^{ v_{\theta}} \right)^{ \overline{k}}
    \left( \overline{\xi}  \right) 
    \right|_{\xi_{\gamma}=1}
    ,
    \\
    \left.   
    \left( \tau_{\beta\gamma}^{ v_{\theta}} \right)^{ \overline{k}}
    \left( \overline{\xi}  \right) 
    \right|_{\xi_{\gamma}=0}
    =
    \left.   
    \left( \tau_{\beta\gamma}^{ v_{\theta}} \right)^{ \overline{k}}
    \left( \overline{\xi}  \right) 
    \right|_{\xi_{\gamma}=1},
    \\
    \beta,\gamma \in \{x,y,z\} 
    \end{gathered}
\end{equation}

Из равенства 
(\ref{elhp:eq27})
следует, что к полученным равенствам следует добавить нормировки на 
ячейковые функции вектора перемещения

\begin{equation}
    \label{elhp:eq31}
    \left< \left( U_{\beta}^{v_{\theta}} \right)^{ \overline{k} } \left( \overline{\xi}  \right)   \right> = 0
    ,\;
    \gamma \in \{x,y,z\} ,\; k \ge 1
\end{equation}


Необходимое условие разрешимости краевой задачи 
(\ref{elhp:eq28})
-
(\ref{elhp:eq31})
имеет вид

\begin{equation}
    \label{elhp:eq32}
    \left( B_{\gamma}^{v_{\theta}} \right)^{ \overline{k} } =
    \left< 
    \left( \tau_{ \gamma x}^{v_{\theta}} \right)^{ \overline{k} - \overline{\epsilon_x} } +
    \left( \tau_{ \gamma y}^{v_{\theta}} \right)^{ \overline{k} - \overline{\epsilon_y} } +
    \left( \tau_{ \gamma z}^{v_{\theta}} \right)^{ \overline{k} - \overline{\epsilon_z} } 
    \right> 
    ,\;
    \gamma \in \{x,y,z\} 
\end{equation}

где треугольные скобки обозначают интегрирование по объёму перидической ячейки (по ячейковым переменным). Из формул 
(\ref{elhp:eq29})
и 
(\ref{elhp:eq32})
следует, что коэффициент 
$\left( B_{\gamma}^{v_{\theta}} \right)^{ \overline{k} }$ 
является функцией от констат 
$A_{v_{\theta}}^{ \overline{k} - \epsilon_y} ,\;A_{v_{\theta}}^{ \overline{k} - \epsilon_y} ,\;A_{v_{\theta}}^{ \overline{k} - \epsilon_z}$. 
Равенство 
(\ref{elhp:eq26})
можно рассматривать как систему уравнений в частныхпроизводных на 
макроперемещения 
$v_{\gamma}^{(n)}$:

\begin{equation}
    \label{elhp:eq33}
    \sum_{ \theta \in \{x,y,z\} } \sum_{k=2}^{n+1}
    \left( 
        \sum_{k_x+k_y+k_z = k}
        \left( B_{\gamma}^{v_{\theta}} \right)^{ \overline{k} }
        \frac{\partial^k v_{\theta}^{(n)}}{\partial \overline{r}^{ \overline{k}}}
        \varepsilon^k
    \right) 
    + f_{ \gamma} \left( \overline{r}  \right) = 0
    ,\;
    \gamma \in \{x,y,z\} 
\end{equation}

Для которой в качестве краевых условий выступаеют условия исходной краевой задачи 
(\ref{elhp:eq16})
-
(\ref{elhp:eq19})
. Равенство 
(\ref{elhp:eq25})
в общем случае не может быть выполнено 
в точности, поэтому потребуем его выполнения в смысле минимизации невязки по методу наменьших квадратов, тогда постановка задачи которую следует
решить, принимает вид.

Это задача о минимизации функционала

\begin{equation}
    \label{elhp:eq34}
    \int_V \left( 
    T_0 \left( \overline{r}  \right) - 
    \sum_{ \theta \in \{x,y,z\} } \left( 
        \sum^{n-1}_{k=1} A_{v_{\theta}}^{ \overline{k} } \frac{\partial^k v_{\theta}}{\partial \overline{r}^{ \overline{k} } } 
    \right)   \right)^2
    dV \rightarrow min
\end{equation}

при выполнении условий

\begin{equation}
    \label{elhp:eq35}
    \sum_{ \theta \in \{x,y,z\} } \sum_{k=2}^{n}
    \left( 
        \sum_{k_x+k_y+k_z = k}
        \left( B_{\gamma}^{v_{\theta}} \right)^{ \overline{k} }
        \frac{\partial^k v_{\theta}^{(n)}}{\partial \overline{r}^{ \overline{k}}}
        \varepsilon^k
    \right) 
    + f_{ \gamma} \left( \overline{r}  \right) = 0
    ,\;
    \gamma \in \{x,y,z\} 
\end{equation}

и граничных учловий на поверхности тела, где коэффициенты 
$\left( B_{\gamma}^{v_{\theta}} \right)^{ \overline{k} }$ 
вычисляются по формулам 
(\ref{elhp:eq32})
на основе решений ячейкрвых краевых задач 
(\ref{elhp:eq28})
-
(\ref{elhp:eq31})
.
Наибольший интерес в любом ассимаптотическом методе представляет первое приближение при 
$n=1$
, поэтому в дальнейшем будем рассматривать
только его.

Первое асимптотическое приближение при 
$n=1$
. В этом случае задача 
(\ref{elhp:eq34})
-
(\ref{elhp:eq35})
принимает вид:

\begin{equation}
    \label{elhp:eq36}
    \int_V \left( 
    T_0 \left( \overline{r}  \right) - 
    \sum_{ \theta \in \{x,y,z\} } \left( 
        \sum_{ \eta \in \{x,y,z\} } A_{v_{\theta}}^{ \overline{k} } \frac{\partial^k v_{\theta}}{\partial \eta } 
    \right)   \right)^2
    dV \rightarrow min
\end{equation}

при выполнении условий

\begin{equation}
    \label{elhp:eq37}
    \sum_{ \theta \in \{x,y,z\} } 
    \left( 
        \sum_{k_x+k_y+k_z = 2}
        \left( B_{\gamma}^{v_{\theta}} \right)^{ \overline{k} }
        \frac{\partial^k v_{\theta}^{(n)}}{\partial \overline{r}^{ \overline{k}}}
        \varepsilon^2
    \right) 
    + f_{ \gamma} \left( \overline{r}  \right) = 0
    ,\;
    \gamma \in \{x,y,z\} 
\end{equation}

и граничных условий на поверхности тела; коэффициенты
$ \left( B_{\alpha}^{ v_{\theta}} \right)^{ \overline{k}} $
вычисляются по формулам
(\ref{elhp:eq32})
на основе решений ячейковых краевых задач 
(\ref{elhp:eq28})
-
(\ref{elhp:eq31})
Ячейковая задача 
(\ref{elhp:eq27})
-
(\ref{elhp:eq21})
при $k=0$ в соответствии с (\colorbox{yellow}{СССЫЛКО!!!}) имеет следующее решение

\begin{equation}
    \label{elhp:eq38}
    \left( U_{\alpha}^{v_{\theta}} \right)^{ \overline{0} } = \delta_{\alpha}^{\theta}
    ,\;
    \left( \tau_{\alpha\beta}^{v_{\theta}} \right)^{ \overline{0} } = 0
    ,\;
    \alpha,\beta,\theta \in \{x,y,z\} 
\end{equation}

Краевая задача на ячейке при 
$k=1$
. Система уравнений равновесия на ячейке

\begin{equation}
    \label{elhp:eq39}
    \frac{\partial \left(  \tau_{\alpha x}^{v_{\theta}}\right) }{\partial \xi_x} +
    \frac{\partial \left(  \tau_{\alpha y}^{v_{\theta}}\right) }{\partial \xi_y} +
    \frac{\partial \left(  \tau_{\alpha z}^{v_{\theta}}\right) }{\partial \xi_z} = 0
    ,\;
    \alpha \in \{x,y,z\} 
\end{equation}

закон термоупругости внутри ячейки

\begin{equation}
    \label{elhp:eq40}
    \left( \tau_{\alpha\beta}^{v_{\theta}} \right)^{ \overline{\epsilon}_{\lambda} } =
    E_{\alpha\beta\theta\lambda} +
    \sum_{ \phi,\psi \in \{x,y,z\} } E_{\alpha\beta\phi\psi}
    \left( 
        \frac{\partial \left( U_{\phi}^{v_{\theta}} \right) }{\partial \xi_{\psi}} - \alpha_{\phi\psi} \Psi A_{v_{\theta}}^{ \overline{\epsilon}_{\lambda}} 
    \right) 
\end{equation}

условия непрерывности ячейковых функций внутри ячейки на границе различных сред

\begin{equation}
    \label{elhp:eq41}
    \left[ \left( \tau_{\alpha n}^{v_{\theta}} \right)^{ \overline{\epsilon}_{\lambda} }  \right] = 0
    ,\;
    \left[ \left( U_{\alpha}^{v_{\theta}} \right)^{ \overline{\epsilon}_{\lambda} }  \right] = 0
    ,\;
    , \; \alpha \in \{x,y,z\}
\end{equation}

условие периодичности ячейковых функций

\begin{equation}
    \label{elhp:eq42}
    \begin{gathered}
    \left.  \left( U_{\alpha}^{v_{\theta}} \right)^{ \overline{\epsilon}_{\lambda}} \left( \overline{\xi}  \right)   \right|_{\xi_{\gamma} = 0}
    =
    \left.  \left( U_{\alpha}^{v_{\theta}} \right)^{ \overline{\epsilon}_{\lambda}} \left( \overline{\xi}  \right)   \right|_{\xi_{\gamma} = 1}
    , \\
    \left.  \left( \tau_{\alpha \gamma}^{v_{\theta}} \right)^{ \overline{\epsilon}_{\lambda}} \left( \overline{\xi}  \right)   \right|_{\xi_{\gamma} = 0}
    =
    \left.  \left( \tau_{\alpha \gamma}^{v_{\theta}} \right)^{ \overline{\epsilon}_{\lambda}} \left( \overline{\xi}  \right)   \right|_{\xi_{\gamma} = 1}
    , \\
    \alpha, \gamma \in \{x,y,z\} 
    \end{gathered}
\end{equation}

условие нормировки решения

\begin{equation}
    \label{elhp:eq43}
    \left< \left( U_{\alpha}^{v_{\theta}} \right)^{ \overline{\epsilon}_{\lambda}}  \right> = 0 ,\; , \; \alpha \in \{x,y,z\}
\end{equation}

Представим ячейковые функции, являющиеся решениями краевой задачи 
(\ref{elhp:eq39})
-
(\ref{elhp:eq43})
, в виде:

\begin{equation}
    \label{elhp:eq44}
    \left( U_{\alpha}^{v_{\theta}} \right)^{ \overline{\epsilon}_{\lambda}} =
    \left( U_{\alpha}^{v_{\theta}} \right)^{ \overline{\epsilon}_{\lambda}}_E +
    A_{v_{\theta}}^{ \overline{\epsilon}_{\lambda}}
    U_{\alpha}^T
    ,\;
    \left( \tau_{\alpha\beta}^{v_{\theta}} \right)^{ \overline{\epsilon}_{\lambda}} =
    \left( \tau_{\alpha\beta}^{v_{\theta}} \right)^{ \overline{\epsilon}_{\lambda}}_E +
    A_{v_{\theta}}^{ \overline{\epsilon}_{\lambda}}
    \tau_{\alpha\beta}^T
\end{equation}

и подставим их в краевую задачу 
(\ref{elhp:eq39})
-
(\ref{elhp:eq43})
, тогда легко видеть, что если считать, что для функций 
$\left( \tau_{ \alpha\beta}^{ v_{\theta}} \right)_E^{ \overline{e}_{\lambda}}$
выполняется равенства

\begin{equation}
    \label{elhp:eq45}
    \left( \tau_{\alpha\beta}^{\nu_{\theta}} \right)_E^{ \overline{e}_{\lambda} }  =
    E_{\alpha\beta\theta\lambda} + \sum_{ \phi,\psi \in \{x,y,z\} } 
    E_{\alpha\beta\phi\psi} \frac{ \partial \left( U_{\phi}^{\nu_{\theta}} \right)_E^{\overline{e}_{\lambda}} }
    { \partial \xi_{\psi}}
\end{equation}

а для функции 
$\tau_{ \alpha\beta}^T$

\begin{equation}
    \label{elhp:eq46}
    \tau_{ \alpha\beta}^T = \sum_{ \phi\psi \in \{x,y,z\}} E_{ \alpha\beta \phi\psi}
    \left( \frac{ \partial U_{\phi}^T}{ \partial \xi_{psi}} - \alpha_{ \phi\psi} \Psi \right) 
\end{equation}

то функции  
$\left( \tau_{ \alpha\beta}^{ v_{\theta}} \right)_E^{ \overline{e}_{\lambda}}$
и 
$\left( U_{ \phi}^{ v_{\theta}} \right)_E^{ \overline{e}_{\lambda}}$
удовлетворяют краевой задаче 
(\ref{elhp:eq39})
-
(\ref{elhp:eq43})
где вместо формул 
(\ref{elhp:eq40})
стоят формулы

(\ref{elhp:eq45})
, а функции 
$\tau_{ \alpha\beta}^T$
и
$U_{ \phi}^T$
удовлетворяют краевой задаче 
(\ref{elhp:eq39})
-
(\ref{elhp:eq43})
, где вместо формул 
(\ref{elhp:eq40})
стоят
формулы 
(\ref{elhp:eq46})
. Для функций 
$\left( \tau_{ \alpha\beta}^{ v_{\theta}} \right)_E^{ \overline{e}_{\lambda}}$
и 
$\left( U_{ \phi}^{ v_{\theta}} \right)_E^{ \overline{e}_{\lambda}}$
в качастве обозначения выбраны нижние индексы E (elasticity), т.к. ячейковые
краевая краевая задача 
(\ref{elhp:eq39})
, 
(\ref{elhp:eq45})
, 
(\ref{elhp:eq41})
-
(\ref{elhp:eq43})
совпадает с ячейковой краевой задачей, 
возникающей при использовании метода ячейковых функций применительно к чисто
упругой задаче без учета изменения температуры \colorbox{yellow}{СССЫЛКО!!!}. Для функций
$\tau_{ \alpha\beta}^T$
и
$U_{ \phi}^T$
в качестве обозначения выбран верхний индекс T (temperature) т.к. коэффтцентам 
$ {A}_{ \overline{e}_{\lambda}}^{ v_{\theta}} $
при этих функциях является нунулевым только при наличии изменений температуры

Расмотрим вычисление констант 
$ \left( B_{\alpha}^{ v_{\theta}} \right)^{ \overline{k}} $
при k=2. Целочисленный вектор 
$ \overline{k}$
представим в виде:

\begin{equation}
    \label{elhp:eq47}
    \overline{k} = \overline{e}_{\lambda} + \overline{e}_{\eta}, \\ \lambda,\eta \in \{x,y,z\}  
\end{equation}

Для таких 
$ \overline{k}$
запишем формулы 
(\ref{elhp:eq32})
:

\begin{equation}
    \label{elhp:eq48}
    \left( B_{\alpha}^{\nu_{\theta}} \right)^{ \overline{e}_{\lambda} + \overline{e}_{\eta}  } 
    =
    \left< 
    \left( \tau_{\alpha x}^{v_{\theta}} \right)^{\overline{e}_{\lambda} + \overline{e}_{\eta} - \overline{e}_x } 
    +
    \left( \tau_{\alpha y}^{v_{\theta}} \right)^{\overline{e}_{\lambda} + \overline{e}_{\eta} - \overline{e}_y } 
    +
    \left( \tau_{\alpha z}^{v_{\theta}} \right)^{\overline{e}_{\lambda} + \overline{e}_{\eta} - \overline{e}_z } 
    \right> 
\end{equation}

Поставим формулы 
(\ref{elhp:eq1}!\colorbox{yellow}{!!!!!!!!!!!!!!!!11})
в формулы 
(\ref{elhp:eq1}!\colorbox{yellow}{!!!!!!!!!!!!!!!!!!!!!!1т})
и приведём подобые при константах 
${A}_{ \overline{e}_{\lambda} + \overline{e}_{\eta} - \overline{e}_z}^{ v_{\theta}} $
:

\begin{equation}
    \label{elhp:eq49}
    \left( B_{\alpha}^{v_{\theta}} \right)^{ \overline{e}_{\lambda} + \overline{e}_{\eta}  } 
    =
    \left( B_{\alpha}^{v_{\theta}} \right)^{ \overline{e}_{\lambda} + \overline{e}_{\eta}  }_E
    -
    A_{v_{\theta}}^{\overline{e}_{\lambda} + \overline{e}_{\eta} - \overline{e}_x }
    \widetilde{\zeta}_{\alpha x}
    -
    A_{v_{\theta}}^{\overline{e}_{\lambda} + \overline{e}_{\eta} - \overline{e}_y }
    \widetilde{\zeta}_{\alpha y}
    -
    A_{v_{\theta}}^{\overline{e}_{\lambda} + \overline{e}_{\eta} - \overline{e}_z }
    \widetilde{\zeta}_{\alpha z}
\end{equation}

где константы 
$ \left( B_{\alpha}^{ v_{\theta}} \right)_E^{ \overline{k}} $
, 
${\zeta}_{\alpha x}$ 
, 
${\zeta}_{\alpha y}$ 
,
${\zeta}_{\alpha z}$ 
вычисляются по формулам:

\begin{equation}
    \label{elhp:eq50}
    \left( B_{\alpha}^{v_{\theta}} \right)^{ \overline{e}_{\lambda} + \overline{e}_{\eta}  }_E
    =
    \left< 
    \left( \tau_{\alpha x}^{v_{\theta}} \right)_E^{\overline{e}_{\lambda} + \overline{e}_{\eta} - \overline{e}_x } 
    +
    \left( \tau_{\alpha y}^{v_{\theta}} \right)_E^{\overline{e}_{\lambda} + \overline{e}_{\eta} - \overline{e}_y } 
    +
    \left( \tau_{\alpha z}^{v_{\theta}} \right)_E^{\overline{e}_{\lambda} + \overline{e}_{\eta} - \overline{e}_z } 
    +
    \left( \tau_{\alpha z}^{v_{\theta}} \right)^{\overline{e}_{\lambda} + \overline{e}_{\eta} - \overline{e}_z } 
    \right> 
\end{equation}

\begin{equation}
    \label{elhp:eq51}
    \widetilde{\zeta}_{\beta \gamma} = 
    - \left< \tau_{\beta \gamma}^T \right> , \beta,\gamma \in \{x,y,z\} 
\end{equation}

В работе \colorbox{yellow}{ССЫЛКО!!!} показано, что упругие характеристики являются усреднёнными ячейковых
функций (в дальнейшем для обозначения макрохарактеристик будем использовать верхнюю
титул):

\begin{equation}
    \label{elhp:eq52}
    \widetilde{E}_{ \alpha\beta \theta\lambda} =
    \left< \left( \tau_{ \alpha\beta}^{ v_{\theta}} \right)_E^{ \overline{e}_{\lambda}}  \right> 
\end{equation}

поэтому величины 
$ \left( B_{\alpha}^{ v_{\theta}} \right)_E^{ \overline{e}_{\lambda} - \overline{e}_{\eta}} $
выражаются через упругие макрохарактеристики 
(\ref{elhp:eq52})
.

Из симметрии констант 
$E_{\beta\gamma \phi\psi}$
по первому и второму индексам и из формул 
(\ref{elhp:eq46})
и 
(\ref{elhp:eq51})
следует симметрия констант 
$ \widetilde{\zeta}_{\beta\gamma}$
по нижнем индексам:

\begin{equation}
    \label{elhp:eq53}
    \widetilde{\zeta}_{\beta\gamma} 
    =
    \widetilde{\zeta}_{\gamma\beta} 
\end{equation}

Для констант 
$ \widetilde{\zeta}_{\beta\gamma}$
выбрано обозначение, использующее тильду, т.к. в дальнейшем станет
понятно, что они обладают определенным физическим смыслом, они образуют собой тензор
второго порядка, который равен свертке тензора макро упругих констант на тензор макро
коэффициентов температурного расширения. Таким образом, эти константы являются
характеристиками макросреды, а для обозначения величин, связанных с макросредой, мы
всегда используем титлу.

Закон термоупругости для однородной макросреды. Для n=1 формула для
напряжений имеет вид

\begin{equation}
    \label{elhp:eq54}
    \sigma_{\gamma\beta} =
    \sum_{ \phi \in \{x,y,z\} } 
    \left( 
        \sum_{ \lambda \in \{x,y,z\} }
        \left( \tau_{\gamma\beta}^{v_{\phi}} \right)^{ \overline{e}_{\lambda}} 
        \frac{ \partial v_{\phi}}{ \partial \lambda} \varepsilon
        +
        \sum_{ k_x+k_y+k_z = 2 }
        \left( \tau_{\gamma\beta}^{v_{\phi}} \right)^{ \overline{k}} 
        \frac{ \partial^2 v_{\phi}}{ \partial \overline{r}^{ \overline{k} } } \varepsilon^2
    \right) 
\end{equation}

Усредним по периодической ячейке компоненты тензора напряжений 
(\ref{elhp:eq54})
и
рассмотрим величины содержащие только первые степени малого параметра 
$\varepsilon$
, введем для
них обозначения такие же как и для компонент тензора напряжений, но содержащие сверху
знак титлы:

\begin{equation}
    \label{elhp:eq55}
    \widetilde{\sigma_{\gamma\beta}} =
    \sum_{ \phi \in \{x,y,z\} } 
    \left( 
        \sum_{ \lambda \in \{x,y,z\} }
        \left< 
        \left( \tau_{\gamma\beta}^{v_{\phi}} \right)^{ \overline{e}_{\lambda}} 
        \right> 
        \frac{ \partial v_{\phi}}{ \partial \lambda} \varepsilon
    \right) 
\end{equation}

Введенные обозначения показывают, что данные величины являются первым асиптотическим
приближением к средним напряжениям 3d-периодической среды. Далее
покажемб что их значения сами по себе могут трактоваться как компоненты тензора
напряженийб возникающих в однородной термоупругой средеб или говоря иначеб 
$ \widetilde{\sigma}_{\gamma\beta}$
в 
уравнения равновесия 
(\ref{elhp:eq8})
и проверим равняется ли левая часть нулю:

\begin{equation}
    \label{elhp:eq56}
    \begin{gathered}
        \sum_{\beta} \frac{ \partial \widetilde{\sigma}_{ \alpha\beta}} { \partial \beta}
            \varepsilon + f_{\alpha} \left( \overline{r}  \right) =
            \sum_{ \beta,\phi \in \{x,y,z\} }
    \left( 
        \sum_{ \lambda \in \{x,y,z\} }
        \left< 
        \left( \tau_{\alpha\beta}^{v_{\phi}} \right)^{ \overline{e}_{\lambda}} 
        \right> 
        \frac{ \partial^2 v_{\phi}}
        { \partial \overline{r}^{ \overline{e}_{\lambda} + \overline{e}_{\beta}} } \varepsilon^2
    \right) 
    + f_{\alpha} \left( \overline{r}  \right)
    = \\
            \sum_{\phi \in \{x,y,z\} }
    \left( 
        \sum_{ \lambda, \eta \in \{x,y,z\} }
        \left( 
            \sum{ \beta \in \{x,y,z\} }
        \left< 
        \left( \tau_{\alpha\beta}^{v_{\phi}} \right)^{\overline{e}_{\lambda} + \overline{e}_{\eta} - \overline{e}_{\beta}} 
        \right> 
        \right) 
        \frac{ \partial^2 v_{\phi}}
        { \partial \overline{r}^{ \overline{e}_{\lambda} + \overline{e}_{\eta}} } \varepsilon^2
    \right) 
    + f_{\alpha} \left( \overline{r}  \right)
    = \\
            \sum_{\phi \in \{x,y,z\} }
    \left( 
        \sum_{ \lambda, \eta \in \{x,y,z\} }
        \left( B_{\alpha}^{ v_{\theta}} \right)_E^{ \overline{e}_{\lambda} + \overline{e}_{\eta}} 
        \frac{ \partial^2 v_{\phi}}
        { \partial \overline{r}^{ \overline{e}_{\lambda} + \overline{e}_{\eta}} } \varepsilon^2
    \right) 
    + f_{\alpha} \left( \overline{r}  \right)
    \end{gathered}
\end{equation}

последнее равенство следует из уравнения 
(\ref{elhp:eq37})
.
Таким образом мы получили равенство:

\begin{equation}
    \label{elhp:eq57}
        \sum_{\beta} \frac{ \partial \widetilde{\sigma}_{ \alpha\beta}} { \partial \beta}
            \varepsilon + f_{\alpha} \left( \overline{r}  \right) = 0
            , \; \alpha \in \{x,y,z\}
\end{equation}

Равенство 
(\ref{elhp:eq57})
означает, что макронапряжения 
$ \widetilde{\sigma}_{ \alpha\beta}$
, подобно компонентам действительного
тензора напряжений, удовлетворяют уравнениям равновесия для некоторой однородной
средый, которую мы будем называть макросредой, и на которую действуют объемные
макросилы 
$f_{\alpha} \left( \overline{r}  \right) $
. Для макросреды в соответствии с формулами Коши может быть введен
тензор деформаций, вычисляемый по ее макроперемещениям:

\begin{equation}
    \label{elhp:eq58}
    \widetilde{e}_{ \alpha\beta} =
    \frac{1}{2} 
    \left( 
        \frac{ \partial v_{\alpha}}{ \partial \beta} +
        \frac{ \partial v_{\beta}}{ \partial \alpha}
    \right) 
\end{equation}

Тогда равенство для макронапряжений 
(\ref{elhp:eq55})
с учетом формул 
(\ref{elhp:eq43})
, 
(\ref{elhp:eq52})
, 
(\ref{elhp:eq55})
может быть
переписано в следующем виде:

\begin{equation}
    \label{elhp:eq59}
    \widetilde{\sigma}_{ \alpha\beta} =
    \sum_{ \theta,\lambda \in \{x,y,z\} }
    \widetilde{E}_{ \alpha\beta \theta\lambda} 
    \widetilde{e}_{ \theta\lambda} -
    \widetilde{\zeta}_{ \alpha\beta} 
    T_0 \left( \overline{r}  \right) 
\end{equation}

Введем тензор второго порядка
$ \widetilde{\alpha}_{\theta\lambda}$
, который по определению удовлетворяет равенству:

\begin{equation}
    \label{elhp:eq60}
    \sum_{ \theta,\lambda \in \{x,y,z\} }
    \widetilde{E}_{ \alpha\beta \theta\lambda}
    \widetilde{\alpha}_{ \theta\lambda}^{\psi} =
    \widetilde{\zeta}_{ \alpha\beta}
\end{equation}

Этот тензор определён однозначно в силу обратимости тензора макроуругих констант 
$ \widetilde{E}_{ \alpha\beta \theta\lambda}$
и являктся симметричным, в силу симметрий тензоров 
$\ \widetilde{\zeta}_{ \alpha\beta}$
и
$ \widetilde{E}_{ \alpha\beta \theta\lambda}$

(\ref{elhp:eq1}!\colorbox{yellow}{!!!!!!!!!!!!!!!!!!1})
:

\begin{equation}
    \label{elhp:eq61}
    \widetilde{\alpha}_{ \theta\lambda} =
    \widetilde{\alpha}_{ \lambda\theta}
\end{equation}

Этот тензор в дальнейшем будем называть 
$\xi$
-тензором температурного расширения макро
термо-упругой среды, т.к. равенство 
(\ref{elhp:eq59})
с помощью формулы 
(\ref{elhp:eq60})
превращается в закон
Неймана-Дюамеля для макро термо-упругой среды:

\begin{equation}
    \label{elhp:eq62}
    \widetilde{\sigma}_{ \alpha\beta} =
    \sum_{ \theta,\lambda \in \{x,y,z\} }
    \widetilde{E}_{ \alpha\beta \theta\lambda}
    \left( 
    \widetilde{e}_{ \theta\lambda} -
    \widetilde{\alpha}_{ \theta\lambda}^{\psi} 
    T_0 \left( \overline{r}  \right) 
    \right) 
\end{equation}

Верхний индекс 
$\xi$
в обозначении коэффициентов 
$ \widetilde{\alpha}_{ \theta\lambda}^{\xi}$
используется для того, чтобы
подчеркнуть их зависимость от распределения температуры внутри ячейки, по этой же
причине тензор тмпературного расширения 
$ \widetilde{\alpha}_{ \theta\lambda}^{\xi}$
имеет смысл именовать 
$\xi$
-тензором
температурного расширения.

Таким образом задача 
(\ref{elhp:eq34})
-
(\ref{elhp:eq35})
о нахождении коэффициентов 
$ {A}_{ \overline{e}_{\lambda}}^{ v_{\theta}} $
может быть сформулирована иначе

\begin{equation}
    \label{elhp:eq63}
    \int_V
    \left( 
        T_0 \left( \overline{r}  \right) -
        \sum_{ \theta \in \{x,y,z\} }
        \left( 
            \sum_{ \eta \in \{x,y,z\} }
            A_{v_{\theta}}^{ \overline{e}_{\eta}}
            \frac{ \partial v_{\theta}}{ \partial \eta}
            \varepsilon
        \right)^2
    \right) 
    dV \rightarrow min
\end{equation}

при выполнении условий

\begin{equation}
    \label{elhp:eq64}
        \sum_{\beta} \frac{ \partial \widetilde{\sigma}_{ \alpha\beta}} { \partial \beta}
            \varepsilon + f_{\alpha} \left( \overline{r}  \right) = 0
            , \; \alpha \in \{x,y,z\}
\end{equation}

а также при выполнении уравнения 
(\ref{elhp:eq62})
и граничных условий на поверхности тела;
коэффициенты 
$ \widetilde{\alpha}_{ \theta\lambda}^{\xi}$
вычисляются по формулам 
(\ref{elhp:eq51})
, 
(\ref{elhp:eq60})
на основе решений ячейковых
краевых задач 
(\ref{elhp:eq39})
, 
(\ref{elhp:eq46})
, 
(\ref{elhp:eq41})
-
(\ref{elhp:eq43})
. Далее ячейковые функции находятся по формулам 
(\ref{elhp:eq44})
,
а напряжения по формулам 
(\ref{elhp:eq54})
.
