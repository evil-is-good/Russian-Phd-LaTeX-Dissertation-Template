Пусть имеется конструкция с границей РРР, произвольной формы, заполненная объёмом РРР. 
Область РРР содержит множество непересекающихся подобластей РРР с поверхностями РРР.
Область
РРР
заполненна однородным материаллом, именуется матрицей или связующим.
Области
РРР
заполнены однородным материалом с иными свойствами, именуются включениями.
Включения расположены периодически в объёме.
Следоваптельно объём РРР заполнен двухфазным периодическим композитным материалом.
Граница 
РРР
является границей раздела двух сред.

Характерный размер включений много больше молекулярно-кинетических размеров.
Наибольший период РРР повторения структуры крмпозита много меньше характерного размера конструкции РРР.
Благодаря первому условию, на микроуровне (на уровне размеров включения) выполняются соотношения механики (пример: закон Гука, закон Фурье).
Материалы входящие в состав композита обладают физическими свойствами (упругость, теплопроводность), экспериментально определяемые на стандартных образцах.
